\documentclass{article}
%\documentclass[final,single]{article}
\addtolength{\oddsidemargin}{-.2in}
\addtolength{\evensidemargin}{-.2in}
\addtolength{\textwidth}{.4in}

%\usepackage{ametsoc}
\usepackage[intlimits]{amsmath}
\usepackage{amssymb,latexsym,graphicx}
\usepackage{lscape}
\usepackage{units}
\usepackage{natbib}  % all options handled by \bibstyle@XXX below
\usepackage{booktabs}
\usepackage[notcite,notref]{showkeys}
\usepackage[osf]{mathpazo} % palatino fonts
\usepackage[linkcolor=blue,citecolor=black,colorlinks=true]{hyperref}
\usepackage{todonotes}
\usepackage{siunitx}
\sisetup{detect-all}
% additional SI unis
\DeclareSIUnit\individual{individual}
\DeclareSIUnit\einstein{E}
\DeclareSIUnit\year{year}

\title{Fast time-scale processes in the Perfect 
Plasticity Approximation (PPA) vegetation model}

% common notation
\newcommand{\wl}{w_l}
\newcommand{\ws}{w_s}
\newcommand{\wlk}{w_{l,k}}
\newcommand{\wsk}{w_{s,k}}
\newcommand{\LAI}{\ensuremath{\text{LAI}}}
\newcommand{\NPP}{\ensuremath{\text{NPP}}}
% derivatives
\newcommand{\pderiv}[2]{\frac{\partial #1}{\partial #2}} % partial derivative
\newcommand{\incr}[2]{\frac{\partial #1}{\partial #2}\Delta #2} 
\newcommand{\iincr}[2]{\Delta #2\frac{\partial #1}{\partial #2}}
\newcommand{\ppderiv}[2]{\Big(\pderiv{#1}{#2}\Big)}   % parenthesized partial derivative 
\newcommand{\fderiv}[2]{\frac{d #1}{d #2}} % full derivative

\newcommand{\nindivs}{n}         % density of individuals in cohort
\newcommand{\crownArea}{a}       % croen area of individual
\newcommand{\csum}{\sum_{k=1}^N} % sum over all cohorts
\newcommand{\layerfrac}{f}       % cohort crown fraction
\newcommand{\intercept}{\gamma}  % interceptef fraction of precip
\newcommand{\avgIntercept}{\overline{\gamma}}

\newcommand{\uptakeFracMax}{f^r}
\newcommand{\uptakeFrac}{f^u}
\newcommand{\dfr}{D_{fr}}
\newcommand{\resScaler}{C_r}
\newcommand{\transpiration}{E_t}

\newcommand{\srl}{\lambda}% specific root length
\newcommand{\psisat}{\psi_*}% saturation (air entry) water potential
\newcommand{\psilow}{\psi_{m}}
\newcommand{\qsat}{\ensuremath{q^\ast}}

% chemical species
\newcommand{\carbon}{\ensuremath{\mathrm{C}}}
\newcommand{\oxigen}{\ensuremath{\mathrm{O}}}
\newcommand{\otwo}{\ensuremath{\mathrm{O_2}}}
\newcommand{\cotwo}{\ensuremath{\mathrm{CO_2}}}
\newcommand{\water}{\ensuremath{\mathrm{H_2O}}}

% photosynthesis
\newcommand{\gsbar}{\ensuremath{\overline{g_s}}}
\newcommand{\anbar}{\ensuremath{\overline{A_n}}}


\begin{document}
\maketitle
\pagestyle{headings}

\numberwithin{equation}{section}

% ===========================================================================
\section{Mass and energy balance equations}

To represent the sub-grid scale heterogeneity of the land surface, each land
grid cell can be split in a number of tiles, with each tile having distinct
physical and biological properties, and its own exchange with the atmosphere.
For example, one tile may represent natural vegetation, another -- cropland, and
yet another -- secondary vegetation that was last disturbed certain amount of
time ago. The energy and mass exchange is calculated separately for each tile,
and the fluxes are aggregated to the grid cell level in the atmosphere.

In the presented manuscript, we only consider one tile (natural vegetation).

Within each tile the vegetation is represented by $N$ cohorts arranged in $L$
layers. Cohorts are numbered from the tallest to the shortest; layers are numbered from
top of the canopy to the bottom. Each cohort $k$ belongs to one layer
(that is, cohorts do not straddle the boundaries between layers). Canopy of
cohort $k$ occupies fraction $\layerfrac_k$ of its layer area. Sum of all
$\layerfrac_k$ is equal to 1 for each layer --- that is, the cohort canopies
fill the entire layer area:
\begin{equation}
   \sum_{k \in L} \layerfrac_k \equiv 1
\end{equation}

Each cohort $k$ is composed of identical individuals, with density $\nindivs_k$
individuals per unit area of tile. Each individual in the cohort has crown area
$\crownArea_k$. We assume that the completely fill the layer space entirely. Since
there is no guarantee that the sum of all crown areas (determined by allometric
relationships) in the layer is going to be equal to the tile area, we need to
stretch canopies of at least some layers. The fraction of the tile area covered
by the canopy of $k$-th cohort is:
%
\begin{equation}
   \label{cohort-frac}
   \layerfrac_k = \nindivs_k \crownArea_k \times
     \big(\sum_{i \in L}\nindivs_i\crownArea_i\big)^{-1}
\end{equation}
%
where the summation is done over all cohorts that belong to layer $L$. 
The leaf area index (\LAI) of each cohort needs to be calculated to preserve
total area of cohort's leaves per unit tile area: $\nindivs_k A_k = \LAI_k
\layerfrac_k$, where $A_k$ is the total one-sided area of leaves per individual.

\begin{equation}
   \LAI_k = \frac{\nindivs_k A_k}{\layerfrac_k}
         = \frac{A_k}{\crownArea_k}\sum_{i \in L}\nindivs_i\crownArea_i
\end{equation}

Each cohorts's canopy has its own temperature $T_v$, and amounts of intercepted water
$\wl$ and snow $\ws$. All cohorts exchange water, energy, and carbon dioxide with the common 
canopy air space of mass $m_c$, temperature $T_c$ and specific 
humidity $q_c$.


With these assumptions, the energy balance of the canopy $k$ can be expressed as:
%
\begin{align}\label{vegn-eb}
   \fderiv{\mathbb{C}_k T_{v,k}}{t} = R_{Sv,k} + R_{Lv,k} + 
        \mathbb{H}_{v,k} - \mathbb{L}_{v,k} - L_f M_{i,k}
\end{align}
where $\mathbb{C}_k$ is the total heat capacity of the canopy, 
$R_{Sv,k}$ and $R_{Lv,k}$ are the net short-wave and long-wave radiative balances of the $k$-th cohort canopy,
$\mathbb{H}_{v,k}$ is the total sensible heat balance of the canopy, 
$\mathbb{L}_{v,k}$ is the total latent heat loss by the canopy,
and
$L_f M_{i,k}$ is the heat associated with the phase transitions of the intercepted water.

The total heat capacity of the cohort canopy $\mathbb{C}_k$ is sum of heat
capacities of leaves $C_{v,k}$, intercepted water $\wlk$, and intercepted snow
$\wsk$:
\begin{equation}
  \mathbb{C}_k = C_{v,k} + \wlk c_l + \wsk c_s
\end{equation}
%
$c_l$ and $c_s$ are the specific heat capacities of the liquid and frozen water, respectively,

Net radiative balances $R_{Sv,k}$ and $R_{Lv,k}$ are calculated using two-stream
approximation, with the assumption of spherical leaf angular distribution~\cite{Pinty-Lavergne-etal-2006}.

The sensible heat term $\mathbb{H}_{v,k}$ in equation~\eqref{vegn-eb} is:
%
\begin{equation}
   \mathbb{H}_{v,k} = - H_{v,k} + H_{pl,i} \intercept_{il,k} + H_{ps,i} \intercept_{is,k} 
   - c_l T_{v,k} D_{l,k} - c_s T_{v,k} D_{s,k}
\end{equation}
%
where $H_{v,k}$ is the sensible heat flux due to turbulent exchange with the canopy air,
$H_{pl,i}$ and $H_{ps,i}$ are the fluxes of heat carried by liquid and solid precipitation,
intercepted by the $k$-th cohort canopy with coefficients $\intercept_{il,k}$,
$\intercept_{is,k}$, 
and $D_{l,k}$, $D_{s,k}$ are the rates of water and snow drip from the canopy.

The latent heat term:
\begin{equation}
   \mathbb{L}_{v,k} =  E_{t,k} L_e(T_{u,k}) + E_{l,k} L_e(T_{v,k}) + E_{s,k} L_s(T_{v,k})
\end{equation}
%
where
where
$E_{t,k}$ is transpiration, 
$E_{l,k}$ -- evaporation of liquid intercepted water,
and
$E_{s,k}$ -- sublimation of intercepted snow.
$L_e(T)$ and $L_s(T)$ are the temperature-dependent specific heats of
evaporation and sublimation, respectively.
$T_{u,k}$ is the temperature of the water the cohort uptakes from the soil.

Note that the terms of equation~\eqref{vegn-eb} are calculated per unit area of
the stretched canopy. These units are convenient for the energy balance calculations,
especially for radiative transport in the multy-cohohrt canopy.

Cohort liquid and solid water balance, respectively:
%
\begin{align}
   \label{vegn-mb-l}
   \fderiv{\wlk}{t} & = \intercept_{il,k} P_{l,i} - D_{l,k} - E_{l,k} + M_{i,k} \\
   \label{vegn-mb-f}
   \fderiv{\wsk}{t} & = \intercept_{is,k} P_{s,i} - D_{s,k} - E_{s,k} - M_{i,k}
\end{align}
%
where $P_{l,i}$ and $P_{s,i}$ are liquid ans solid precipitation on top of layer $i$,
$\intercept_{il,k}$ and $\intercept_{is,k}$ are the fractions of liquid and solid
precipitation that cohort $k$ intecepts,
$D_{l,k}$ and $D_{s,k}$ are the rates of water and snow drip from the canopy,
and
$M_{i,k}$ is the rate of snow melt on the canopy of $k$-th cohort.

On top of the canopy $P_l^1$ is just the rainfall rate from the atmosphere, and
$P_s^1$ is the snowfall rate. 
For layers $i>1$ we can write: 
%
\begin{align}
   P_{l,i} & = \sum_{k \in i-1} \layerfrac_k (1-\intercept_{il,k})P_{l,i-1} \label{drip-l} \\
   P_{s,i} & = \sum_{k \in i-1} \layerfrac_k (1-\intercept_{is,k})P_{s,i-1} \label{drip-s} 
\end{align}
%
where the summation is done over all cohorts in layer $i-1$.
Here we assume that the the drip from the canopy never gets intercepted by the
layers below, and contributes directly to the water and energy balance of the underlying 
surface.

With these assumptions, canopy air water (specific humidity) balance equation is:
%
\begin{equation}\label{cana-mb}
   m_c\fderiv{q_c}{t} = \csum \layerfrac_k E_{v,k} + E_g - E_a
\end{equation}
%
where 
$E_g$ is water vapor flux from the ground surface,
$E_a$ is the water vapor flux to the atmosphere.
The total water vapor flux $E_{v,k}$ from $k$-th cohort canopy to the canopy 
air is a sum of three components:
\begin{equation}
   E_{v,k} \equiv E_{t,k} + E_{l,k} + E_{s,k} 
\end{equation}

Canopy air energy balance:
\begin{align}\label{cana-eb}
   m_c\fderiv{}{t}\big((1-q_c)c_p T_c + q_c c_v T_c\big)
      & = \csum  \layerfrac_k H_{v,k} + H_g - H_a
   \notag \\
      & + c_v \Big(\csum  \layerfrac_k T_{v,k} E_{v,k} + T_g E_g - T_c E_a\Big)
\end{align}
%
where
$c_p$ and $c_v$ are specific heats of dry air and water vapor, respectively,
$H_{v,k}$ is the flux of sensible heat from $k$-th cohort canopy to the canopy air,
$H_g$ is the sensible heat flux from the ground surface to the canopy air,
$H_a$ is the sensible heat flux from canopy air to the atmosphere.

Energy balance of the ground surface:
%
\begin{equation}\label{grnd-eb}
   R_{Sg}+R_{Lg}-H_g - L_g E_g - G - L_f M_g = 0
\end{equation}

\section{Non-water-stressed photosynthesis}

We assume that the whole canopy is isothermal with temperature $T_v$, and the
air in the leaf interior is saturated with specific humidity equal to $q^*(T_v)$.
However, the radiation absorbed by the leaves is not distributed uniformly
through the entire depth of the canopy, with upper parts of the canopy getting
more light and the lower parts being shadowed by the upper ones. Therefore the
net photosynthesis  $A_n$ and, accordingly, stomatal conductance for water vapor
$g_s$ are going to depend on vertical coordinate. Since the dependency of photosynthesis
on light level is highly nonlinear, it is necessary to consider the vertical
profile of $A_n$ and $g_s$ to get average canopy values, rather then base the
calculations on average available radiation. 

To begin with, let's consider a thin layer of the canopy that receives flux $Q$
of the photosynthetically-active incident radiation per unit of leaf area.
Obviously, $Q$ will depend on the vertical coordinate, according to the
radiation absorption and scattering relationships in the canopy.

The link between stomatal conductance $g_s$ \si{\mole\of\water\per\meter\squared\per\second}, 
the rate of net photosynthesis $A_n$, \si{\mole\of\cotwo\per\meter\squared\per\second},
intercellular concentration of \cotwo\ $C_i$,
\si{\mole\of\cotwo\per\mole\of{air}} and difference in specific humidities in
stomata and specific humidity in canopy air $q_a$,
\si{\kilogram\of\water\per\kilogram\of{air}} can be expressed as:
%
\begin{equation}\label{E:Leuning}
   g_s = \frac{m A_n}{(C_i-\Gamma_*)(1+(\qsat(T_v)-q_a)/d_0)}
\end{equation}
%
where $m$ is the slope of the stomatal conductance relationship,
\si{\mole\of\water\per\mole\of{air}},
$d_0$ is a reference value of a water vapor deficit, 
\si{\kilogram\of\water\per\kilogram\of{air}},
$\Gamma_*$ (\si{\mole\of\cotwo\per\mole\of{air}}) is the \cotwo\ compensation 
point:
%
\begin{equation}
   \Gamma_* = \alpha_c[\otwo]\frac{K_\carbon}{2 K_\oxigen}
\end{equation}
%
$\alpha_c = 0.21$ is the maximum ratio of oxygenation to carboxylation, $[\otwo]
= \SI{0.209}{\mole\of\otwo\per\mole\of{air}}$ is the concentration of oxigen in
canopy air, and $K_\carbon$ (\si{\mole\of\cotwo\per\mole\of{air}}) and
$K_\oxigen$ (\si{\mole\of\otwo\per\mole\of{air}}) are the Michaelis-Menten
constants for \cotwo \ and \otwo, respectively. $K_\carbon$ and $K_\oxigen$
depend on temperature proportionally to Arrhenius function:
%
\begin{equation}\label{E:Arrhenius}
   f_A(E_0,T) = \exp\left[E_{0,X}\left(\frac{1}{\SI{288.2}{\kelvin}}-\frac{1}{T}\right)\right]
\end{equation}
%
so that
%
\begin{align}
  K_\carbon &= D_\carbon f_A(E_{0,\carbon},T_v) \\
  K_\oxigen &= D_\oxigen f_A(E_{0,\oxigen},T_v)  
\end{align}
%
with respective constants: 
$D_\carbon = \SI{1.5e-4}{\mole\of\cotwo\per\mole\of{air}}$,
$E_{0,\carbon} = \SI{6000}{\kelvin}$, 
$D_\oxigen = \SI{0.25}{\mole\of\otwo\per\mole\of{air}}$,
and
$E_{0,\oxigen} = \SI{1400}{\kelvin}$.

The equation \eqref{E:Leuning} is a simplification of \cite{1995-Leuning} 
empirical relationship in the assumption of negligible cuticular conductance.

On the other hand, net photosynthesis $A_n$ can also be expressed as a carbon
dioxide diffusive flux between canopy air and the stomata space:
\begin{equation}\label{E:cotwodiff}
   A_n = \frac{g_s}{1.6}(C_a-C_i)
\end{equation}
where $C_a$ is the concentration of \cotwo\ in the canopy air; the factor 1.6 is
the ratio of diffusivities for water vapor and \cotwo. We assume that the
diffusion of \cotwo\ is mostly limited by stomatal conductance and not by the leaf
boundary layer conductance.

Combining equations \eqref{E:Leuning} and \eqref{E:cotwodiff}, we can get an
expression for intercellular concentration of \cotwo:
\begin{equation}
  C_i = \frac{
       C_a+\Gamma_*\frac{1.6}{m}\left( 1+\frac{\qsat(T_v)-q_a}{d_0} \right)
     }{
       1+\frac{1.6}{m}\left( 1+\frac{\qsat(T_v)-q_a}{d_0} \right)
     }
\end{equation}

The mechanistic model of photosynthesis \cite{Farquhar-etal-1980}, with
extensions introduced in \cite{Collatz-etal-1991,Collatz-etal-1992}   expresses
net photosynthesis as a difference between gross photosynthesis and leaf
respiration. Furthermore, the gross photosynthesis is expressed as a minimum of
several physiological process rates:
%
\begin{equation}\label{E:an}
   A_n = f_T(T_v) \begin{cases}
      \min(J_{E,C3}, J_C, J_j)     - \gamma V_m(T_v) & \text{for C3 plants}, \\
      \min(J_{E,C4}, J_C, J_{CO2}) - \gamma V_m(T_v) & \text{for C4 plants}.
   \end{cases}
\end{equation}
%
where 
$f_T(T_v)$ is thermal inhibition factor,
$J_E$ is the light limited rate, 
$J_C$ is the Rubisco limited rate,  
$J_j$ is the export limited rate of carboxylation, 
$J_C$ is a \cotwo -limited rate,
and $V_m(T_v)$ is the maximum velocity of carboxylase,
\si{\mole\of\cotwo\per\meter\squared\per\second}; the term $\gamma V_m(T_v)$
in both cases of equation \eqref{E:an} represents leaf respiration. 

The thermal inhibition factor is expressed as
%
\begin{equation}\label{e:thermal-inhibition}
   f_T(T_v) = 
   \frac{1}{
      \left[1+\exp(0.4(\SI{5}{\celsius} - T_v))\right]\,
      \left[1+\exp(0.4(T_v - \SI{45}{\celsius}))\right]
   }
\end{equation}
%
and affects carbon acquisition and respiration equally.

The maximum velocity of carboxylase $V_m$ depends on the temperature of the 
leaf and on the leaf age, \cite{Wilson-Baldocchi-etal-2001a}:
%
\begin{equation}
  V_m(T_v) = V_{\max} f_A(E_V,T_v) \exp\Big(-\max\Big[\frac{t-t_0}{\tau},0\Big]\Big)
\end{equation}
%
where $V_{\max}$ is species-dependent constant, 
$t$ is the age of leaf, $t_0$ is the time of leaf aging onset, and $\tau$ is 
aging rate.
Note that in the configuration described in this manuscript the aging is only
applied to temperate deciduous trees.

For C3 plants, \cite{Collatz-etal-1991}:
\begin{subequations}\label{E:c3limits}
\begin{align}
   J_E &= a \alpha Q \frac{C_i - \Gamma_*}{C_i+2\Gamma_*} \\
   J_C &= V_m(T_v) \frac{C_i -\Gamma_*}{C_i + K_\carbon(T_v)\frac{p_{ref}}{p}
        \left(
            1+\frac{p}{p_{ref}}\frac{[O_2]}{K_\oxigen(T_v)}
	\right)
      } \\
   J_j &= \frac{V_m(T_v)}{2}
\end{align}
\end{subequations}
%
where
$a$ is the leaf absorptance of photosynthetically-active radiation (PAR), 
$Q$ is incident PAR, \si{\einstein\per\meter\squared\per\second},
%
\footnote{\SI{1}{E}(einstein)=\SI{6.022e23}{photons}, regardless of the photon
energy. In other words, one einstein is defined as one \si{mole} of photons.}
%
$\alpha$ is intrinsic quantum efficiency of photosynthesis,
    \si{\mole\of\cotwo\per\einstein}, 
$p$ is atmospheric pressure and 
$p_{ref}=\SI{1e5}{\pascal}$ is the reference atmospheric pressure.

For C4 plants, \cite{Collatz-etal-1992}:
%
\begin{subequations}\label{E:c4limits}
\begin{align}
   J_E &= a \alpha Q\\
   J_C &= V_m(T_v)\\
   J_{CO2} &= 18000 V_m(T_v) C_i 
\end{align}
\end{subequations}

While the solution of the equations \eqref{E:an}-\eqref{E:c4limits} gives a
photosynthesis rate for a thin canopy layer, given PAR flux $Q$ incident to this
layer, what we actually need is an average photosynthesis for the entire canopy.
Assuming that PAR flux $Q=Q(L)$ monotonically decreases with canopy depth $L$
(defined by equation~\eqref{canopy-depth-definition} on
page~\pageref{canopy-depth-definition}), we can define a layer $L_{eq}$ where
the light-limited rate $J_E$ is equal to the minimum of other limiting rates.
The photosynthesis below the level $L_{eq}$ will be a function of light
availability, while above this level $A_n$ will be function of other limiting
rates. The net photosynthesis averaged over the entire canopy depth can be
expressed as
%
\begin{equation}\label{E:anbar}
   \anbar = \frac{f_T(T_v)}{\LAI}\Bigg[
      J_{\min} L_{eq} +
      \int_{L_{eq}}^{\LAI}J_E(L)\,dL 
   \Bigg]
   -f_T(T_v)\gamma V_m(T_v)
\end{equation}
%
where
%
\begin{equation}
   J_{\min} = \begin{cases}
      \min(J_C, J_j)      & \text{for C3 plants}, \\
      \min(J_C, J_{CO2})  & \text{for C4 plants}.
   \end{cases}
\end{equation}

Assuming that the dependence of light on canopy depth can be described by 
Beer-Lambert-Bouguer law:
\begin{equation}\label{E:bouger}
   Q(L)=Q_0\exp(-\kappa L)
\end{equation}
we can obtain the following expressions for the integral in the right-hand side of
\eqref{E:anbar}:
%
\begin{equation}
  \int_{L_{eq}}^{\LAI}J_E(L)\,dL =
  a \alpha' Q_0\frac{\exp(-\kappa\,L_{eq})-\exp(-\kappa\,\LAI)}{\kappa}
\end{equation}
%
and for $L_{eq}$:
%
\begin{equation}
   L_{eq} = \frac{1}{\kappa}\log\left(\frac{a \alpha' Q_0}{J_{\min}}\right)
\end{equation} 
%
where
%
\begin{equation}\label{E:aprime}
   \alpha' = \begin{cases}
      \alpha \frac{C_i - \Gamma_*}{C_i+2\Gamma_*} & \text{for C3 plants}, \\
      \alpha & \text{for C4 plants}.
   \end{cases}
\end{equation}

Average stomatal conductance is calculated from \eqref{E:anbar} similar to 
equation~\eqref{E:Leuning}
\begin{equation}
   \gsbar = \begin{cases}
      \frac{m \anbar}{(C_i-\Gamma_*)(1+(\qsat(T_v)-q_a)/d_0)} & \anbar > 0 \\
      g_{s,\min} & \anbar \leq 0
   \end{cases}
\end{equation}
%
where $g_{s,min} = \SI{0.01}{\mole\of\water\per\meter\squared\per\second}$
is the minimum stomatal conductance allowed in the model.

The model applies some further corrections to the calculated net photosynthesis
and stomatal conductance. If there is water or snow on the canopy, the
photosynthesis is reduced proportionally to the covered fraction of leaves:
%
\begin{align}
   \anbar & = \anbar\,(1-(f_s+f_w)\alpha_\text{wet}) \\
   \gsbar & = \gsbar\,(1-(f_s+f_w)\alpha_\text{wet})
\end{align}
%
where $f_l$ and $f_s$ are the fractions of canopy covered by liquid water and
snow, respectively;
$\alpha_\text{wet}$ is the down-regulation coefficient assumed to be equal 0.3.
That means that the photosynthesis of the leaf fully covered by water or snow
will be reduced by 30\% compared to the dry leaf.

The model then imposes an upper limit on the value of stomatal conductance: if the
calculated $\gsbar$ is higher than the limit $g_s^{\max} =
\SI{0.25}{\mole\of\water\per\meter\squared\per\second}$, then the stomatal
conductance and net photosynthesis are adjusted:
%
\begin{align}
   \anbar & = \anbar \times \begin{cases}
          {g_s^{\max}}/{\gsbar}, &  \anbar > 0 \\
          1, & \anbar \le 0
       \end{cases} \\
   \gsbar & = g_s^{\max}
\end{align}

% - - - - - - - - - - - - - - - - - - - - - - - - - - - - - - - - - - - - - - -
\section{Effect of water on photosynthesis}

Last, but not least, the stomatal conductance and photosynthesis need to be
adjusted for available water limitations. Given stomatal conductance,
the water demand per unit area of land, \si{\mole\of\water\per\meter\squared\per\second} 
can be calculated as
%
\begin{equation}
   U_d = \gsbar \, \LAI \, (q^*(T_v)-q_a) \frac{M_{\text{air}}}{M_\water}
\end{equation}
%
where the factor $M_{\text{air}}/M_\water$ is used to convert water vapor 
deficit to units of \si{\mole\per\mole}, to be compatible with the units of 
\gsbar.

Given the maximum water supply rate $U_{\max}$, the net photosynthesis and
stomatal conductance are adjusted as:
%
\begin{align}
   \anbar & = \anbar \times \min(U_{\max}/U_d,1) \\
   \gsbar & = \gsbar \times \min(U_{\max}/U_d,1)
\end{align}

The maximum soil-controlled uptake by the root is defined as that with root water
potential at the plant permanent wilting point. To calculate this maximum rate
of water uptake we use the standard 2D radial flow model formulation
\cite{Darcy-1856a}, in the quasi-steady flow approximation.

Let 
$u$ be the water uptake rate per unit length of fine root, \si{\kilogram\per\meter\per\second},
$R$ -- characteristic radial half-distance to the next root, \si{\meter}, 
$r_r$ -- root radius, \si{\meter}, 
and 
$r$ -- "microscopic"  distance from root axis, \si{\meter}.

For steady flow toward the root,
\begin{equation}
   u = 2 \pi r K \frac{d \psi}{d r}
\end{equation}
where $K=K(\psi)$ is unsaturated hydraulic conductivity \si{\kilogram\per\meter\squared\per\second}, 
\begin{equation}\label{K:soil}
   K(\psi) = \begin{cases}
      \smash{K_s \left(\frac{\psi}{\psisat}\right)^{-(2+3/b)}} & \psi \leq \psisat \\
      K_s  & \psi > \psisat
      \end{cases}
\end{equation}
where 
$\psi$ is the soil water matric head, \si{\meter}, and 
$\psisat$ is the air entry water potential. 
Note that since the flow is assumed to be steady-state, $u$ doesn't depend on $r$. 

Integrating from root-soil interface to "bulk" soil (with matric head $\psi_s$ at
the distance  $R$ from the root axis, and $\psi_r$ at the root surface):
\begin{equation}\label{u0}
   \int_{r_r}^R \frac{u dr}{2 \pi r} = \int_{\psi_r}^{\psi_s} K(\psi) d\psi
\end{equation}
or, equivalently:
\begin{equation}\label{u1}
   u = \frac{2 \pi}{\ln(R/r_r)} \int_{\psi_r}^{\psi_s} K(\psi) d\psi
\end{equation}
This relationship is assumed to hold at a macroscopic point, i.e., a model layer
in our case. The integral in the righ-hand side of equation~\eqref{u0} is
sometimes called \emph{matric flux potential}, \cite{Raats-2007a}.

On the other hand, the water flux through the root skin per unit length of root
is
%
\begin{equation}\label{u:root}
   u = 2 \pi r_r K_r(\psi_r - \psi_x)
\end{equation}
%
where $K_r$ is permeability of root membrane per unit membrane area, 
\si{\kilogram\per\cubic\meter\per\second},
and $\psi_x$ is the water potential inside the  root (xylem water potential) 
\si{\meter}.

The characteristic half-distance between roots $R$ can be expressed in terms of
total root length per unit volume of soil. Suppoze that cohort $k$ has the
specific root length $\lambda_k$, \si{\meter\per\kilogram\of{C}} (SRL, length of
fine roots per unit mass of carbon) and the volumetric density of root biomass
$b_{r,k}$, \si{\kilogram\of{C}\per\cubic\meter}. The total length of roots of all
cohorts per unit volume is $\sum\nindivs_k\lambda_k b_{r,k}$; therefore the area
of soil cross-section surrounding the root is $A=\pi
R^2=1/\sum\nindivs_k\lambda_k b_{r,k}$, giving
\begin{equation}
  R=\Big(\pi \sum\nindivs_k \lambda_k b_{r,k}\Big)^{-1/2}
\end{equation}

Combining \eqref{u1}, \eqref{K:soil}, and \eqref{u:root}, after tedious algebraic
transformations we get:
\begin{multline}\label{u:final}
   r_r K_r(\psi_r - \psi_x)=\frac{2 \pi K_s}{\ln(R/r_r)}\Bigg\{
        \frac{\psisat}{n} \bigg[
           \left(\frac{\min(\psi_s,\psisat)}{\psisat}\right)^n -
           \left(\frac{\min(\psi_r,\psisat)}{\psisat}\right)^n 
        \bigg]\\
      +\max(0,\psi_s-\psisat) - \max(0,\psi_r-\psisat)
   \Bigg\}
\end{multline}
%
where we introduced notation $n=-(1+3/b)$.

Given xylem water potential $\psi_x$ and soil water potential $\psi_s$, we can
solve the equation \eqref{u:final} to get the water potential  at the root-soil
interface $\psi_r$, and, consequently, the water flux per unit root length
$u=u(\psi_r,\psi_s)$. 

To calculate the total water uptake, we should note that the xylem potential
increases with depth so that $\psi_x = \psi_{x0}+z$, where $\psi_{x0}$ is the 
xylem potential at the surface. 
The total uptake will be then the sum of layer values, properly weighted:
\begin{equation}\label{u:total}
   U(\psi_{x0}) = \sum_{1}^{N} u(\psi_{x0}+z_i, \psi_i) L_i S_i
\end{equation}
where $z_i$ is the depth of the 
layer, $\psi_i$ is the soil water potential in the layer, and $L_i$ is the total 
length of roots in the layer. The additional factor $S_i$ is used to turn off uptake
when certain conditions are met: when there is ice in the layer; when
the uptake is negative (optional, when one-way-uptake is requested); and when
the soil is saturated (optional, when uptake-from-sat is not requested).

The maximum soil water supply to the vegetation $U_{\max}$ is calculated as the value 
of the uptake for the xylem water potential at the surface equal to the permanent 
wilting point $\psi_{wilt}$: $U_{\max}=U(\psi_{wilt})$.

\bibliographystyle{ametsoc}
\bibliography{literature}

\end{document}